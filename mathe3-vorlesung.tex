\documentclass[a4paper]{scrartcl}

\usepackage[utf8]{inputenc}
\usepackage[T1]{fontenc}
\usepackage[ngerman]{babel}
\usepackage{amsmath}
\usepackage{amssymb}

\begin{document}

\section{Grundlagen über reelle Zahlen}

\subsection{Geordnete Körper}

\paragraph{Definition 1}
\begin{tabbing}
\begin{tabular}{l}
Es sei A eine Menge. Eine Relation $ \prec \subseteq  AxA $heißt totale Ordnung.\\
1) $ \prec $ ist eine Ordnungsrelation (reflexiv, antisymmetrisch, transitiv)\\
2) $ \forall $ a,b $ \in $ A $ \lbrace $ a $ \prec $ b oder b $ \prec $ a $ \rbrace $
\end{tabular}
\end{tabbing}

\paragraph{Bezeichnung 1}
\begin{tabbing}
\begin{tabular}{l}
Ist a $ \prec $ b und a $ \neq $ b, so schreiben wir a $ \prec $ b.
\end{tabular}
\end{tabbing}

\paragraph{Beispiel 1}
\begin{tabbing}
\begin{tabular}{l}
Die natürliche Ordnung $ \leq $ der reellen Zahlen, ist eine totale Ordnung.\\
Sind a,b $ \in \mathbb{R} $ mit a < b, dann a+c < b+c für alle c $ \in \mathbb{R}$\\
(a=5, b=6, c=2, 5<6, 5+2 < 6+2, d.h. 7<8)\\
0<a,b $ \Rightarrow $ 0 < a*b\\
(a = 5, b = 6, 0 < 5*6 = 30)
\end{tabular}
\end{tabbing}

\paragraph{Definition 2}
\begin{tabbing}
\begin{tabular}{l}
Es sei (K,+,*) ein Körper mit Nullelement 0 und $ \prec $ eine totale Ordnung auf K. Dann heißt $ \prec $ Anordnung wenn : \\
1) $ \forall x,y,z \in K (y<z \Rightarrow x+y < x+z) $ \\
2) $ \forall x,y,z \in K (0<x,y \Rightarrow 0<x*z) $\\
$ (K,+,*,\prec) $ heißt geordneter Körper.
\end{tabular}
\end{tabbing}

\paragraph{Beispiel 2}
\begin{tabbing}
\begin{tabular}{l}
$( \mathbb{R} ,+,*, \prec ) $ heißt geordneter Körper.
\end{tabular}
\end{tabbing}

\paragraph{Satz 1}
\begin{tabbing}
\begin{tabular}{l}
Es sei $ (K,+,*,\prec) $ ein geordneter Körper.\\
Dann gilt $ 0\prec a \; oder \; a=0 \; oder \; 0\prec -a \; fuer \; a \in K $
\end{tabular}
\end{tabbing}

\paragraph{Beweis (Satz 1)}
\begin{tabbing}
\begin{tabular}{l}
Angenommen $ a \prec 0 $ und $ a \neq 0 $.\\
Dann $ a \prec 0  $ nach Definition 1 und $ a-a \prec -a $, d.h.
$ 0 \prec -a $ nach Definition 2.
\end{tabular}
\end{tabbing}

\paragraph{Satz 2}
\begin{tabbing}
\begin{tabular}{l}
Es sei $ (K,+,*,\preceq) $ ein geordneter Körper und $ a,b,c,d \in K $.\\
Ist $ a \preceq b\; und\; c \preceq d,\; dann\; a+c \preceq b+d. $
\end{tabular}
\end{tabbing}

\paragraph{Beweis (Satz 2)}
\begin{tabbing}
\begin{tabular}{l}
Aus $ a\preceq b $ folgt $ a+c \preceq b+c $.\\
Aus $ c\preceq d $ folgt $ b+c \preceq b+d $.\\
Die Transitivität liefert : $ a+c \preceq b+d $
\end{tabular}
\end{tabbing}

\paragraph{Bemerkung 1}
\begin{tabbing}
\begin{tabular}{l}
Ist a=c=0, dann liefert Satz 2 : 
$ (0 \preceq b\; und\; 0 \preceq d) \Rightarrow 0 \preceq b+d. $
\end{tabular}
\end{tabbing}

\paragraph{Definition 3}
\begin{tabbing}
\begin{tabular}{l}
Es sei (K,+,*) ein Körper mit Nullelement D und $P\subseteq K$. \\
P heißt Positivbereich von K, wenn\\
1) Für alle $a\in K\setminus\{ 0\}$ gilt entweder $a\in P$ oder $-a\in P$\\
2) $a,b\in P \Rightarrow a+b, a*b\in P$
\end{tabular}
\end{tabbing}

\paragraph{Bezeichnung 2}
\begin{tabbing}
\begin{tabular}{l}
Wir setzen $P_< := \{ a\in K| 0 \prec a\}$ 
\end{tabular}
\end{tabbing}

\paragraph{Satz 3}
\begin{tabbing}
\begin{tabular}{l}
Ist $(K,+,*,\preceq)$ ein geordneter Körper, dann ist \\
$P_<$ ein Positivbereich.
\end{tabular}
\end{tabbing}

\paragraph{Beweis (Satz 3)}
\begin{tabbing}
\begin{tabular}{l}
1) Es sei $a\in K\setminus \{ 0\}$. Es sei $a \notin P_<$ dann gilt\\
a = 0 oder $0 \prec -a$ nach Satz 1, d.h.\\
$0 \prec -a$ da $a \neq 0$. Somit $(-a) \in P_<$.\\
Ist $a\in P_<$, d.h. $0\prec a$, $-a+0 \prec a-a$.\\
$-a \prec 0$, d.h. $-a\in P_<$. Analog $-a\in P_< \Rightarrow a \notin P_<$.\\
Somit gilt entweder $a\in P_< \; oder \; -a\in P_<$\\
2) Es seien $a,b\in P_<$, d.h. $0 \prec a, 0\prec b$. Nach Bemerkung 1 gilt\\
$0 \prec a+b$ und $a+b\in P_<$. Außerdem gilt $0\prec a*b$ nach Definition 2, d.h. $a*b\in P_<.$
\end{tabular}
\end{tabbing}

\paragraph{Bezeichnung 3}
\begin{tabbing}
\begin{tabular}{l}
Wir setzen $\prec_P = \{ (x,y)| y+x\in P\cup\{ 0\} \}$
\end{tabular}
\end{tabbing}

\paragraph{Satz 4}
\begin{tabbing}
\begin{tabular}{l}
Es sei $(K,+,*)$ ein Körper mit Nullelement 0 und $P \subseteq K$ ein Positivbereich.\\
Dann ist $(K,+,*,\prec_P)$ ein geordneter Körper.
\end{tabular}
\end{tabbing}

\paragraph{Beweis (Satz 4)}
\begin{tabbing}
\begin{tabular}{l}
Zunächst zeigen wir, das $\prec_P$ eine totale Ordnung ist (Definition 1)\\
- reflexiv : Es sei $a\in K$. Dann $a-a=0$, d.h. $a \prec_P a$ \\
- antisymmetrisch : $a \prec_P b$ und $b \prec_P a$, dann $b-a\in P\cup\{ 0\}$ und $a-b\in P\cup \{ 0\}$.\\
Ist $b-a\in P$, dann $-(b-a)\notin P$ (Definition 3), d.h.\\
Ist $b-a\in P$, ergibt sich $b-a=0$ also a-b.\\
-transitiv : Es seien $a\prec_P b$ und $b\prec_P c$, d.h.\\
$b-a\in P\cup\{ 0\}$ und $c-b\in P\cup\{ 0\}$.\\
Dann $c-a=(c-b) + (b-a)\in P\cup\{ 0\}$ nach Definition 3\\
\\
Damit ist $\prec_P$ eine Ordnungsrelation\\
Es seien $a,b\in K.$ ist a=b, dann a-b=0, d.h. $a \prec_P b$.\\
Ist $a \neq b$, d.h. $b-a \neq 0$. Dann gilt entweder $(b-a)\in P,\;-(b-a)\in P.$\\
$a-b\in P$ nach Definition 3\\
d.h. $a \prec_P a$ oder $b \prec_P a$\\
Damit ist $\prec_P$ eine totale Ordnung und wir haben noch 1) und 2) aus Definition 2 zu zeigen.\\
\\
1) Es seien $x,y,z\in K$ mit $y \prec_P z$, d.h. $z-y\in P\cup\{ 0\}$.\\
Dann $y \notin z$ und $z-y\in P$.\\
Dann $(z + x) - (y-x)\in P$, d.h. $y+x \prec_P z+x.$\\
2) Es seien $0 \prec_P x,y$, d.h. $x-0\in P\cup \{ 0\}$ und $y-0\in P\cup\{ 0\}$,\\
also $x,z \in P\cup\{ 0\}$ und somit $x,y\in P$, daraus folgt \\
$x*y\in P$, $x*y-o\in P$ und $0 \prec_P x*y$
\end{tabular}
\end{tabbing}

\paragraph{Bemerkung 2}
\begin{tabbing}
\begin{tabular}{l}
Ist P ein Positivbereich eines geordneten Körpers $(K,+,*,\prec)$.\\
Dann gilt $\prec_P = \prec$
\end{tabular}
\end{tabbing}

\paragraph{Satz 5}
\begin{tabbing}
\begin{tabular}{l}
Es sei $(K,+,*)$ ein Körper mit Nullelement 0 sowie $a\in K\setminus\{ 0\}$ und \\
$P \subseteq K$ ein Positivbereich. Dann gilt $a^2\in P$
\end{tabular}
\end{tabbing}

\paragraph{Beweis (Satz 5)}
\begin{tabbing}
\begin{tabular}{l}
Da $a\neq 0$ gilt $a\in P$ oder $-a\in P$ nach Definition 3.\\
Dann $a^2\in P$ oder $(-a)^2\in P$ (d.h. $a^2\in P$) nach Definition 3.
\end{tabular}
\end{tabbing}

\paragraph{Satz 6}
\begin{tabbing}
\begin{tabular}{l}
Es sei $(K,+,*)$ ein Körper und P ein Positionsbereich von K.\\
Ist $a\in P$ dann $a^{-1}\in P$ (für alle $a\in K\setminus\{ 0\}$)
\end{tabular}
\end{tabbing}

\paragraph{Beweis (Satz 6)}
\begin{tabbing}
\begin{tabular}{l}
Nach Satz 5 gilt $(a^{-1})^2\in P$.\\
Aus $a, (a^{-1})^2\in P$ folgt $a(a^{-1})^2\in P$ nach Definition 3, d.h. $a^{-1}\in P$\\
\end{tabular}
\end{tabbing}

\paragraph{Satz 7}
\begin{tabbing}
\begin{tabular}{l}
Es sei $(K,+,*)$ ein Körper mit Einselement e und $\emptyset\neq P\subseteq K$ ein Positivbereich\\
von K. Dann gilt $e\in P$.
\end{tabular}
\end{tabbing}

\paragraph{Beweis (Satz 7)}
\begin{tabbing}
\begin{tabular}{l}
Nach Satz 5 gilt $e^2-e\in P$
\end{tabular}
\end{tabbing}

\paragraph{Satz 8}
\begin{tabbing}
\begin{tabular}{l}
Es sei $(K,+,*,\prec)$ ein geordneter Körper mit Nullelement 0 sowie $a,b,c\in K$ mit $a\prec b$.\\
Dann gilt :
a) $0\prec c \Rightarrow ca \prec cb$\\
b) $c\prec 0 \Rightarrow cb \prec ca$
\end{tabular}
\end{tabbing}

\paragraph{Beweis (Satz 8)}
\begin{tabbing}
\begin{tabular}{l}
Ist c=0 oder a=b, dann ist die Aussage klar.\\
Es seien nun $c \neq 0$ und $a\neq b$.\\
Aus $a\prec b$ folgt $a-a\prec b-a$, d.h. $0\prec b-a$, d.h. $b-a\in P_<$\\
wobei $P_<$ ein Positivbereich ist (Satz 3)\\
a) $0\prec c$ heißt $c\in P_<$, so $c(b-a)\in P_<$ (Definition 3), d.h.\\
$cb-ca\in P_<$ und $ca \prec_{P_<} cb$, d.h. $ca \prec cb$ nach Bemerkung 2.\\
b) $c\prec 0$, also $c\prec_{P_<} 0$, d.h. $0-c\in P_<$\\
somit $(-c)(b-a)\in P_<$, d.h. $ca-cb\in P_<,\;cb\prec_{P_<} ca$\\
also $cb \prec ca$.
\end{tabular}
\end{tabbing}

\paragraph{Satz 9}
\begin{tabbing}
\begin{tabular}{l}
Es sei $(K,+,*,\preceq)$ ein geordneter Körper und $a,b\in P_<$.\\
Ist $a\prec b$, dann $b^{-1} \prec a^{-1}$
\end{tabular}
\end{tabbing}










\paragraph{Satz 18}
\begin{tabbing}
\begin{tabular}{l}
Es sei M eine Menge und $ \preceq $ eine totale Ordnung auf M. Dann sind die\\
folgenden Eigenschaften äquivalent : \\
i) $ \preceq $ erfüllt die Supremum-Eigenschaft.\\
ii) $ \preceq $ erfüllt die Infimum-Eigenschaft.
\end{tabular}
\end{tabbing}

\paragraph{Beweis (Satz 18) i) $ \Rightarrow $ ii)}
\begin{tabbing}
\begin{tabular}{l}
Es sei $ \emptyset \neq B \subseteq M $ nach unten beschränkt.\\
Wir setzen $ U_B = \{ x \in M | x\;ist\;untere\;Schranke\;von\;B \} $ \\ 
$ U_B  \neq \emptyset $, da B nach unten beschränkt ist.\\
$ U_B $ ist nach oben beschränkt, denn für $ b\in B $ \\
gilt $ x \preceq b $ für alle $ x\in U_B \; Da \preceq $ die \\
Supremum-Eigenschaft erfüllt, existiert $ Sup(U_B)$.\\
Wir zeigen, $Sup(U_B)$ = Inf(B). Es sei $ b\in B $. Da b eine obere Schranke\\
von $U_B$ ist, gilt $Sup(U_B) \preceq $ b.\\
($Sup(U_B)$ ist die kleinste obere Schranke von $ U_B $)\\
Somit ist $Sup(U_B)$ untere Schranke von B.\\
Es sei $ e\in U_B $ und $ u \preceq Sup(U_B) $. Somit\\
ist $Sup(U_B)$ die größte untere Schranke von B, d.h.\\
$Sup(U_B) = Inf(B)$. Also existiert das Infimum von B.\\
i) $ \Rightarrow $ i) .Es sei $ \emptyset \neq B \subseteq M $ und nach \\
oben beschränkt.\\
Wir setzen $ O_B :=\{ x\in M | x\;ist\;obere\;Schranke\;von\;B \}$.\\
Man kann analog zeigen, dass $Inf(O_B)$ existiert und Sup(B) = $Inf(O_B)$. 
\end{tabular}
\end{tabbing}

\paragraph{Definition 10}
\begin{tabbing}
\begin{tabular}{l}
Es sei M eine Menge und $ \preceq $ eine totale Ordnung auf M.\\
$\preceq$ heißt vollständige Ordnung, wenn $\preceq$ die\\
Supremum-Eigenschaft erfüllt.
\end{tabular}
\end{tabbing}

\paragraph{Definition 11}
\begin{tabbing}
\begin{tabular}{l}
ein geordneter Körper $(K,+,*,\preceq )$ heißt vollständig wenn\\
$\preceq$ eine vollständige Ordnung ist.
\end{tabular}
\end{tabbing}

\paragraph{Satz 19}
\begin{tabbing}
\begin{tabular}{l}
Es sei $(K,+,*,\preceq)$ ein vollständiger Körper.\\
Dann ist die Anordnung $\preceq$ archimedisch.
\end{tabular}
\end{tabbing}

\paragraph{Beweis (Satz 19)}
\begin{tabbing}
\begin{tabular}{l}
Es sei $a \in K$. Es sei $a \preceq 0$. Dann gilt \\
$a \preceq 1 \; denn\; 0 \preceq 1$. Es sei $0 \preceq a$.\\
Angenommen, es gibt kein $n \in \mathbb{N} $ mit $a \preceq n$ \\
$ (\exists n\;a \prec n) \sim (\forall n\;n\preceq a)$. Dann $n\preceq a$ \\
für alle $n\in \mathbb{N} $. Dann $ \aleph \subseteq K $ nach oben\\
beschränkt. Da $\preceq$ eine vollständige Ordnung ist, existiert 
Sup($ \mathbb{N} $)\\
Aus $ -1 \prec 0 $ folgt $Sup(\mathbb{N}) -1 \prec Sup(\mathbb{N})$ \\
d.h.$Sup(\mathbb{N})-1$ ist keine obere Schranke von $\mathbb{N} \subseteq K$. \\
Damit existiert ein $m\in \mathbb{N}$ mit $Sup(\mathbb{N})-1 \prec m+1$ \\
$(Sup(\mathbb{N})-1 )+1\prec m+1$ , d.h. $Sup(\mathbb{N})\prec m+1 \in \mathbb{N}$\\
Dies ist ein Widerspruch zu Sup($\mathbb{N}$) ist obere\\
Schranke von $\mathbb{N} \subseteq K$. Damit ist die Annahme falsch und es existiert $n\in \mathbb{N}$ mit a<n.
\end{tabular}
\end{tabbing}

\subsection{Wurzeln}

\paragraph{Definition 12}
\begin{tabbing}
\begin{tabular}{l}
Es sei (K,+,*) ein Körper, $a,y\in K$\\
$ 2 \leq k \in \mathbb{N} $. Ist yk = a, dann heißt yk-te Wurzel aus a.\\
\end{tabular}
\end{tabbing}

\paragraph{Beispiel 12}
\begin{tabbing}
\begin{tabular}{l}
3 und (-3) und 4-te Wurzel aus 81 : $\sqrt[4]{81}$\\
(-3) ist 3-te Wurzel aus -27 : $\sqrt[3]{-27}$
\end{tabular}
\end{tabbing}

\paragraph{Satz 20}
\begin{tabbing}
\begin{tabular}{l}
Es sei $(K,+,*,\preceq)$ ein total geordneter Körper \\
$ n\in \mathbb{N} \setminus \{ 0\} $ und $a\in K$ mit $ 0 \preceq a$.
Dann gibt es höchstens ein $y\in K$ mit $0 \prec y \; und y^n = a$ .\\
\end{tabular}
\end{tabbing}

\paragraph{Beweis (Satz 20)}
\begin{tabbing}
\begin{tabular}{l}
Es seien $y_1,y_2 \in K$ mit $y^n_1 = y^n_2 = a$.\\
Dann $y_1 \preceq y_2$ oder $y_2 \preceq y_1$ oder $ y_1 = y_2$.\\
Es sei $0\prec y_1 \prec y_2$ Wir zeigen mit vollstaendiger Induktion\\
dass $0 \prec <y^n_1 \prec y^n_2$\\
IA : n=1 ist klar nach Voraussetzung $ 0 \prec y_1 \prec y_2 $\\
IV : n=k Es gilt $ 0 \prec y^k_1 \prec y^k_2$.\\
IBh: n=k+1 Es gilt $0 \prec y^{k+1}_1 \prec y^{k+1}_2$\\
IBw: Nach IV gilt $0 \prec y^k_1 \prec y^k_2$ und $0\prec y_1$ \\
Nach Satz 8a) erhaelt man\\
$0 \prec y^{k+1}_1 \prec y^{k+1}_2$ Aus $y_1 \prec y_2$ und\\
$0 \prec y_2$ folgt $ y_1y_2^k \prec y_2^{k+1} $ (Satz 8a). \\
Die Transitivitaet liefert :\\
$0 \prec y_1^{k+1} \prec y_2^ky_1\prec y_2^{k+1}$, d.h.\\
$0\prec y_1^{k+1} \prec y_2^{k+1})$\\
Damit $y_1^n \neq y_2^n$ , ein Widerspruch. Analog erhaelt man\\
ein Widerspruch falls $y_2 \prec y_1$ . Also $y_1 = y_2$.
\end{tabular}
\end{tabbing}

\paragraph{Satz 20.a}
\begin{tabbing}
\begin{tabular}{l}
(O.B.) Es sei (K,+,*,$\preceq$) ein archimedischer Koerper,\\
$n\in \aleph \setminus \{ 0\}$ und $a\in K$ mit $0\prec a$.\\
Dann sind die folgenden Aussagen aequivalent.\\
i) Es existiert ein $y\in K$ mit $0\prec y$ und $y^n=a$\\
ii) Die Menge B :=$\{ x\in K| x^n=a\}$ hat ein supremum.\\
und es gilt $Sup(B)^n=a$ sowie $0\prec sup(B)$.\\
\end{tabular}
\end{tabbing}

\paragraph{Bemerkung 6}
\begin{tabbing}
\begin{tabular}{l}
Ist (K,+,*,$preceq$) ein vollstaendiger Koerper,\\
$a \in K $ mit $0 \prec a$ und $n \in \aleph \setminus \{ 0\}$.\\
Dann gibt es genau ein $y \in K$ mit $0\prec y$ und $y^n = a$.
\end{tabular}
\end{tabbing}

\paragraph{Bezeichnung}
\begin{tabbing}
\begin{tabular}{l}
Man schreibt $ \sqrt[n]{a}$ = y oder y = $a^{1/n}$\\
fuer die n-te Wurzel aus a.
\end{tabular}
\end{tabbing}

\paragraph{Satz 21}
\begin{tabbing}
\begin{tabular}{l}
Es sei (K,+,*) ein Koerper. Dann existiert hoechstens eine totale Ordnung\\
$\preceq$, so dass (K,+,*,$\preceq$) vollstaendig ist. 
\end{tabular}
\end{tabbing}

\paragraph{Beweis (Satz 21)}
\begin{tabbing}
\begin{tabular}{l}
Es sei $\preceq$ eine totale Ordnung, so dass (K,+,*,$\prec$)\\
vollstaendig ist. Es sei $a\in K\setminus \{ 0\}$. Ist a=x fuer\\
$x\in K\setminus \{ 0\}$, dann ist $a=x^2 \in P_<$ nach Satz 5. d.h. $0\prec a$\\
Ist $0 \prec a$, dann existiert ein $x\in K\setminus \{ 0\}$ \\
mit $a=x^2$ nach Bemerkung 6.\\
da (K,+,*,$\preceq$) vollstaendig ist. Dann gilt \\
$0 \prec a \Leftrightarrow x \in K \setminus \{ 0\} mit x^2=a $existiert.\\
Dies zeigt $P_< = \{ x^2|x\in K\setminus \{ 0\} \}$ fuer jede totale Ordnung\\
$preceq$ fuer die (K,+,*,$\preceq$) vollstaendig ist. Nach Bemerkung 2 gilt\\
$ \preceq = \preceq_{P_<} $. Ausserdem gilt \\
$\preceq_{P_<} = \{ (a,b)|b-a \in \{ x^2|x \in K\setminus \{ 0\} \}\cup \{ 0\} \}$\\
Dies zeigt, dass die totale Ordnung eindeutig durch den Koerper (K,+,*)\\
festgelegt ist (falls eine derartige totale Ordnung $\preceq$ existiert)
\end{tabular}
\end{tabbing}

\paragraph{Bemerkung 7}
\begin{tabbing}
\begin{tabular}{l}
Es sei $a\in K$ mit $0 \prec a$ und $m,n\in \aleph \setminus \{ 0\}$.\\
Dann gilt $\sqrt[n]{a^m} = (\sqrt[n]{a})^m$\\
und $\sqrt[n]{\sqrt[m]{a}}=\sqrt[nm]{a}=\sqrt[m]{\sqrt[n]{a}}$
\end{tabular}
\end{tabbing}

\paragraph{Bezeichnung}
\begin{tabbing}
\begin{tabular}{l}
$a^{m/n} = \sqrt[n]{a^m}$
\end{tabular}
\end{tabbing}

\paragraph{Beispiel 13}
\begin{tabbing}
\begin{tabular}{l}
$\sqrt[2]{\sqrt[3]{64}} = \sqrt[2]{4} = 2$\\
$\sqrt[3]{\sqrt[2]{64}}=\sqrt[3]{8}=2$\\
$\sqrt[6]{64}=2$\\
$2^{4/2} = \sqrt[2]{2^4}=\sqrt[2]{16}=4$
\end{tabular}
\end{tabbing}

\paragraph{Bemerkung 8}
\begin{tabbing}
\begin{tabular}{l}
1) Fuer $q\in Q$ mit q<0 und $a\in K$ gilt $a^q=(a^{-q})^{-1}$\\
2) $q\in Q$ und $a,b\in K$ mit $0\prec a,b$ gilt $(ab)^q=a^qb^q$\\
3) Fuer $p,q \in Q$ und $a\in K$ mit $0\prec a$ gilt\\
$(a^p)^q = a^{p*q}$ und $a^p*a^q = a^{p+q}$\\
4) Fuer $q\in Q^+$ und $a,b\in K$ mit $0 \preceq a \prec b$ gilt\\
$a^q\prec b^q$
5) Fuer $q\in Q$ mit $q\prec 0$ und $a,b\in K$ mit\\
$0 \prec a \prec b$ gilt $b^q \prec a^q$\
\end{tabular}
\end{tabbing}

\subsection{Die reellen Zahlen als DedeKind-Schnitte}

\paragraph{Definition 13}
\begin{tabbing}
\begin{tabular}{l}
Es sei $\emptyset\neq M \subset Q$. M heisst DedeKind-Schnitt,\\
wenn fuer alle $p\in M$ gilt $\{ a\in Q|a\preceq p \}\subset M$
\end{tabular}
\end{tabbing}

\paragraph{Bemerkung 9}
\begin{tabbing}
\begin{tabular}{l}
Ist M ein DedeKind-Schnitt dann gilt
a) $\forall p\in M \forall a\in Q \{ a\leq p \Rightarrow a\in M \}$
b) $\forall p\in M \exists r\in M \{ p<r \}$
\end{tabular}
\end{tabbing}

\paragraph{Beispiel 14}
\begin{tabbing}
\begin{tabular}{l}
$M=\{ x\in Q|x^3 < -1 \}$ Wir wollen zeigen, dass M ein DedeKind-Schniit ist\\
Es sei $p\in M$, d.h. $p^3<-1$. Fuer $a\in Q$ mit $a\leq p$ gilt\\
$a^3 \leq p^3 < -1$, d.h. $a^3 < -1$. Dann ist $\{ a\in Q|a \leq p\}\subseteq M$\\
Es gilt $p \neq -1$ und nach Satz 17 existiert ein $r\in Q$ mit\\
$p < r < -1$, d.h. $r^3<-1$ und somit $r\in M$. Dies zeigt\\
$\{ a\in Q|a\subseteq p\}\subset M$, da r nicht zur linken Menge gehoert,\\
aber in M liegt.
\end{tabular}
\end{tabbing}

\paragraph{Satz 22}
\begin{tabbing}
\begin{tabular}{l}
Es sei $M \subseteq Q$ ein Dedekind-Schnitt.\\
Dann ist M eine nach oben beschraenkte Menge in dem Koerper $(Q,+,*,\leq)$\\
\end{tabular}
\end{tabbing}

\paragraph{Beweis (Satz 22)}
\begin{tabbing}
\begin{tabular}{l}
Angenommen, M ist nicht nach oben beschraenkt.\\
.[Dann gilt es zu jedem $m\in \aleph$ ein $q\in M$ mit $m \subseteq q$] *\\
Es sei $q\in Q$. Dann existiert ein $n\in \aleph\setminus\{ 0\}$\\
mit $q\subset n $ (Def 6), da $(Q,+,*,\leq)$ archimedisch ist.\\
Zu n existiert wegen (*) ein $p\in M$ mit $n \leq p$, d.h. $n \in M$, da\\
M Dedekind-Schnitt und wegen q<n ist auch $q\in M$. Also \\
$Q \subseteq M$, das widerspricht $M \neq Q$ (Definition 13)
\end{tabular}
\end{tabbing}

\paragraph{Satz 23}
\begin{tabbing}
\begin{tabular}{l}
(O.B.) Man kann auf der Menge aller Dedekind-Schnitte (bezeichnet mit M*)\\
eine Addition $\oplus$ und eine Multiplikation $\odot$ sowie eine totale\\
Ordnung $\preceq$ definieren, sodass $(M*,\oplus,\odot,\preceq)$\\
ein vollstaendiger Koerper ist, der $(Q,+,*)$ enthaelt und $\preceq$\\
die natuerliche Ordnung $leq$ der rationalen Zahlen fortsetzt.\\
Jeder geordnete Koerper mit dieser Eigenschaft ist isomorph zu\\
$(M*,\oplus,\odot,\preceq)$.
\end{tabular}
\end{tabbing}

\paragraph{Bezeichnung}
\begin{tabbing}
\begin{tabular}{l}
Der in Satz 23 betrachtete Koerper $(M*,\oplus,\odot,\preceq)$\\
wird der Koerper der reellen Zahlen genannt und mit (R,+,*) bezeichnet.
\end{tabular}
\end{tabbing}

\subsection{Absolutbetrag und Bewertung}

\paragraph{Definition 14}
\begin{tabbing}
\begin{tabular}{l}
Es seien $(K,+,*,\preceq)$ ein geordneter Koerper mit Nullelement 0\\
und $a\in K$ . Dann heisst\\
$|a|:=\lbrace a falls 0 \preceq a\; oder\; -a falls a\prec 0$ \\
Absolutbetrag von a
\end{tabular}
\end{tabbing}

\paragraph{Bemerkung}
\begin{tabbing}
\begin{tabular}{l}
Der Absolutbetrag von a ist von der Anordnung $\preceq$ abhaengig.
\end{tabular}
\end{tabbing}

\paragraph{Beispiel 15}
\begin{tabbing}
\begin{tabular}{l}
Ist $r\in \Re$, dann beschreibt |r| bezueglich der\\
natuerlichen Ordnung $\leq$ der reellen Zahlen den Abstand von r\\
zum Nullpunkt auf dem genormten Zahlenstrahl.
\end{tabular}
\end{tabbing}

\paragraph{Bemerkung 11}
\begin{tabbing}
\begin{tabular}{l}
a) Es gilt $|a|=|-a|$\\
b) $|a| \geq 0$\\
c) $|a| = 0 \Leftrightarrow a=0$\\
d) $|a*b| = |a|*|b|$\\
e) $|a+b| \leq |a|+|b| (Dreiecksungleichung)$\\
f) $|a|-|b| \leq |a-b|,|a+b|$\\
g) $||a|-|b|| \leq  |a-b|$\\
\end{tabular}
\end{tabbing}

\paragraph{Bezeichnung}
\begin{tabbing}
\begin{tabular}{l}
i ist ein Symbol mit $i\; \notin)\Re$. Wir setzen $i^2=-1$.\\
Sind $a,b\in\Re$, dann heisst a+b komplexe Zahl.\\
a heisst Realteil, b heisst Imaginaerteil.\\
$C:= \{ a+ib|a,b\in\Re\}$ - MEnge der $|a+ib|:=\sqrt[2]{a^2+b^2}$\\
heisst Norm von a+ib\\
z=a+ib, dann heisst $\overline{z}:=a-ib$ konjugiert komplexe Zahl\\
zu z = a+ib
\end{tabular}
\end{tabbing}

\paragraph{Bemerkung 12}
\begin{tabbing}
\begin{tabular}{l}
a) Es gilt $(a+ib)(a+ib) = aa'-bb'+i(ab'+a'b)$\\
b) $(a+ib)+(a'+ib')=(a+a')+i(b*b')$\\
Ist z=a+ib, dann $z*\overline{z}=a^2+b^2$, d.h. \\
$|z| = \sqrt[2]{z*\overline{z}}$
\end{tabular}
\end{tabbing}

\paragraph{Beispiel 16}
\begin{tabbing}
\begin{tabular}{l}
z = 4+i3  ,  $|z|=\sqrt[2]{4^2+3^2}=\sqrt{25}=5$\\
$z_1 = 4+i3, z_2= 1-i2$\\
$z_1+z_2= 4+1+i(3-2)=5+i$\\
$z_1*z_2= 4*1-(3(-2))+i(4(-2)+3*1)=10+i(-5) = 10-i5$
\end{tabular}
\end{tabbing}

\end{document}
